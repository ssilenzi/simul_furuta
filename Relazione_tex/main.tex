\input{structure.tex}
\graphicspath{{./figs/}}
\pgfplotsset{
	table/search path={./figs},
}

\title{\normalfont{Simulazione e Controllo del Pendolo Inverso di Furuta}}
\author{Francesco Petracci e Simone Silenzi} 


\begin{document}

\input{title.tex}

\section{Indicazioni Report}
Project reportA  report  must  be  produced  (8-10  pages)  to  explain  the  project  details.  In  particular,  the  title  page  must  contain  the  name  of  the  course  under  which  the  project  has  been  done,  the  project  title,  an  optional  picture related  to  the  project,  the  delivery  date,  and the  author(s)  name(s)  with  contact  information (student ID number and Email). The  report  must  include  a  general  description  of  the  project,  the  physical  model  used,  the  design  choices, the user interface, the shared data structures, the tasks involved, a task-resource diagram, a discussion  on  the  task  parameters  (how they  were  defined),  and  a  set  of  experimental  results  that show  the  behavior  of  the  system  as  a  function  of  specific  variables.  Figures  and  screen  shots  are  welcome. The project code must not be included in the report, but in a separate folder.

TO DO LIST:
\begin{itemize}
	\item RIFARE SCHEMA PENDOLO,primo link da rifare il cilindro finale
	\item AGGIORNARE la parte sulla grafica in modo da rispiecchiare drawings.m
	\item Conclusioni
\end{itemize}
 
\section{Introduzione}
Questo progetto consiste nella simulazione dell'interazione real-time tra il pendolo inverso di Furuta, una scheda STM32F4 Discovery e un'interfaccia utente su pc. 

L'applicazione finale \`e stata sviluppata in C sfruttando principalmente le librerie Allegro e Pthread. Per quanto riguarda la simulazione del pendolo e della scheda \`e stato fatto uso del software \textsc{Matlab--Simulink} per sviluppare un modello fisico, il relativo controllo e per quindi creare in modo automatico alcune funzioni.

\section{Modello}
\label{sez:modello}
Da scrivere

\section{Controllo}
\label{sez:controllo}
Da scrivere

\section{Interfaccia}
\label{sez:interfaccia}

\begin{figure}[]
	\centering
	%\def\svgwidth{0.7\linewidth}
	%\input{schema_2.pdf_tex}
	\includegraphics[width=.7\linewidth]{schema_2.pdf}
	\caption{Schema del Pendolo \textbf{DA SISTEMAREE}}
	\label{fig:pendolo_schema}
\end{figure}

\begin{figure}[]
	\centering
	\includegraphics[height=.4\textheight]{interfaccia_in_esecuzione.png}
	\caption{Interfaccia durante l'esecuzione}
	\label{fig:interfaccia_in_esecuzione}
\end{figure}
L'interfaccia \`e divisa in due zone principali: quella di comunicazione con l'utente e le tre viste del pendolo.
Per quanto riguarda la zona di comunicazione questa \`e popolata dalle variabili di interesse e di come modificarle. Si \`e quindi riportato a schermo:
\begin{itemize}
	\item il riferimento su $\alpha$
	
	\item  il vettore di stato $\left[ \alpha \ \theta \ volt\right]^T$
	
	\item i parametri di controllo
	
	\item disturbo, rumore e ritardi
	
	\item polo della funzione di trasferimento che genera il riferimento su $\alpha$
	
	\item periodo del task di controllo
	
	\item per ciascun task il numero di deadline miss, il tempo di esecuzione in $\si{\micro \second}$ e il worst-case execution time in $\si{\micro \second}$
	
	\item varie istruzioni su come resettare e come uscire dal programma
	
\end{itemize}
Per quanto riguarda le viste, queste sono tre. La vista dall'alto evidenzia il primo link e quindi \`e possibile vedere $\alpha$ mentre in quella da lato $\theta$.

Per costruire la vista assonometrica \`e stato utile strutturare il problema in coordinate omogenee e introdurre dei sistemi di riferimento ausiliari. Tenendo d'occhio la figura~\ref{fig:pendolo_schema} e intendendo con $R_i(\varphi)$ la generica rotazione intorno all'asse $i$ di un angolo $\varphi$ e con $T_{ij}$ la trasformazione in coordinate omogenee dal sistema di riferimento $i$ a quello $j$ tale che per un generico vettore $p$ valga: $^ip = T_{ij}\ ^j p$; si pu\`o elaborare i cambi di coordinate necessari. 
\begin{equation}
	^0\! T_{01}(\alpha) = \left[ \begin{array}{ccc|c}
	& & & l_1 \\
	& R_z(\alpha) & & 0 \\
	& & & 0 \\
	\hline
	0 & 0 & 0 & 1
	\end{array} \right]
	\quad 
	^1 T_{12}(\vartheta) =
	\left[  \begin{array}{ccc|c}
	& & & 0 \\
	& R_x(\theta) & & 0 \\
	& & & 0 \\
	\hline
	0 & 0 & 0 & 1
	\end{array}
	\right]
\label{eq:cambio_coordinate}
\end{equation}
Quindi si possono ricavare le coordinate in sistema di riferimento $0$ dei punti di interesse:
\begin{equation}
	\begin{aligned}
		^0\! \mathrm{OA}& =\ ^0\!T_{01}(\alpha) \begin{bmatrix} 0& 0 & 0& 1 \end{bmatrix}^t\label{eq:from_punti_in_coord_0} \\
		^0\! \mathrm{AP}& = \ ^0\! T_{12}(\vartheta)
		\left[  \begin{array}{ccc|c}
		& & & 0 \\
		& Id & & 0 \\
		& & & l_2 \\
		\hline
		0 & 0 & 0 & 1
		\end{array}
		\right] \\
		^0\! \mathrm{OP}& = \ ^0\! \mathrm{OA}+\ ^0\! \mathrm{AP}
	\end{aligned}
\end{equation}
e in forma estesa:
\begin{equation}
	\begin{aligned}
	^0\!\mathrm{OA} = &\left[ \begin{array}{c} l_1 \cos\alpha \\ l_1 \sin \alpha \\ 0 \end{array}\right] \\
	^0\!\mathrm{OP} = &\left[ \begin{array}{c} l_1 \cos \alpha - l_2 \sin \alpha \sin \vartheta\\ l_1 \sin \alpha + l_2 \cos \alpha \sin \vartheta\\ l_2 \cos \vartheta \end{array}\right]
	\end{aligned}
\end{equation}
Per poi ottenere le coordinate di un punto qualsiasi  nell'interfaccia \`e sufficiente proiettare nel piano usando la seguente relazione, indicando con $s$ gli assi assonometrici, con $\rho$ l'angolo di longitudine e con $\lambda$ quello di latitudine:
\begin{equation}
	R_{s0} = R_z (\rho)  R_y (-\lambda)  \begin{bmatrix}
	0 & 0 & -1\\
	1 & 0 & 0\\
	0 & -1 & 0
	\end{bmatrix}
\label{eq:proiezione}
\end{equation}
Il generico punto $^0\!\left[ x \ y \ z \right]^t $ nel piano di disegno di \textsc{Allegro} risulta quindi:

\begin{equation}
	x_{s} = x\sin\rho + y\cos\rho, \quad y_{s} = x \cos \rho \sin \lambda + y \sin \lambda \sin \rho - z \cos \lambda
\end{equation}


\section{Overview del codice}

Il software \`e stato sviluppato in C usando principalmente \textsc{Allegro} 4.4, \textsc{pthread} e l'Embedded Coder di \textsc{Simulink--Matlab}. La cartella principale ha 3 principali sotto--cartelle: \textit{Matlab} contiene i file \textit{Matlab} sviluppati per elaborare il modello fisico e le viste, \textit{include} gli header utilizzati e infine \textit{src} i file sorgente. Una cartella aggiuntiva, \textit{build}, viene creata al momento della compilazione e contiene i file object e il file binario.

Una parte dei file sorgenti e dei file header sono stati generati in modo automatico tramite l'utilizzo appunto dell'Embedded Coder. Questi sono stati contrassegnati ad inizio file da un'intestazione e sono stati inoltre descritti brevemente nella prima parte del file \texttt{main.c}.

\texttt{Altro?}

\subsection{Data structures}
\input{resource.tex}

\subsection{Task}
\input{task.tex}
\FloatBarrier

\section{Conclusioni (?)}
Da Scrivere?

	
\end{document}